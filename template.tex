\documentclass[12pt,twocolumn]{article} 

\usepackage{oxycomps} % use the main oxycomps style file

\bibliography{references}

\pdfinfo{
    /Title (CS Ethics Paper)
    /Author (Eddie Valdez)
}

\title{CS Ethics Paper}

\author{Eddie Valdez}
\affiliation{Occidental College}
\email{evaldez@oxy.edu}

\begin{document}

\maketitle

\begin{introduction}

 An important consideration that must take place in the development of any technology system are the ethical issues that system could produce. Many major technology systems have been designed without considering and addressing ethical issues. That is why these issues will be discussed for this project. While my comps project does not contain an overwhelming amount of ethical concerns, there are several ethical issues that my application could perpetuate even though it is trying to help people out by improving their mental health through mindfulness. These issues my app can perpetuate include data bias, accessibility issues, potentially concentrating power to a few people, technological solutionism, and consent issues. 
 \\
\end{introduction}



\begin{body}


Ethical issues in data bias could exist. I am a straight cisgender middle class male Latino Computer Science major who was raised Catholic and am attending a prestigious private liberal arts college. I am not an expert by any means on cognitive science, psychology, behavior, or mindfulness practices broadly. This could lead to false information. I am aware of my privileges as a male cisgender identity as well as my ethnicity as a Latino person. Because of this, I will try to be actively aware of designing my app to be as unbiased as possible. However, there are definitely  biases that I have that I am not consciously aware of. Because I am the only person who will be designing this app, this could lead to my own biases being coded into the design of my app. In order to minimize this, I plan on having input from community members about the design and functionality of my app.

To this point, Sasha Costanza-Chock in her book design justice talks about how when we design products, we must design them with the community it is intended to help in order to meet their needs and remove our own biases\cite{Chock20}. Chock notes that most designers "do not think of themselves as sexist, racist, homophobic, etc..", however, their own perspectives may inhibit their ability to see how their design choices negatively affect oppressed communities\cite{Chock20}. This means including community members, specifically people who have been oppressed, in the design process and giving them credit for it. This is something that I have not done.However, I would like to include community input to help my app be more useful and rid of biases. So, if I want my app to be more inclusive and free of my own biases I would have to engage community members to be a part of the app’s design so my perspective is not the only one being considered in the design of my app.

My app would have multiple ethical accessibility issues. The first accessibility issue would be that because it will be an iOS app, only people with iPhones, iPads, or iPods would be able to use the app. This means that there are significant accessibility issues as many people do not have these types of devices. This might further perpetuate existing mental health inequalities because people without phones, meaning they are likely of lower socioeconomic status,have a higher likelihood of developing and experiencing mental health problems\cite{WHO14}. This is an issue because this is the audience that I would want my app to help the most. 

The fact that my app will be in English will limit who can use my app. Only those who can read English would be able to find the app potentially useful.


Another accessibility issue includes the fact that my app would not likely be effective for Unaccompanied Migrant Youth. This is because Unaccompanied Migrant Youth themselves are not going to be co-designers of my app. I conclude this because research shows that when Unaccompanied Migrant Youth are not co-designers of mental health technologies and the macro system’s influence is not considered in the design, mental health technologies have been shown to be ineffective\cite{Tachtler21}. Marco system influences include cultural, linguistic and socio-economical factors. So because I am not having Unaccompanied Migrant Youth be a part of my design process, it may not be effective for them. This research study supports Sasha Costanza Chock’s point that the community we are helping should be a part of the design process. Additionally, Sasha Costanza Chock writes about how when we design for those at the margins of society, meaning those who face the most intersecting of oppressed identities, it makes design more practical for everyone\cite{Chock20}. Because of this, by designing my app with Unaccompanied Migrant Youth in consideration, my app will be easier to use for everyone who would use the app. Thus making it a more useful and inclusive app.


Another ethical concern about my app is that it has the power to concentrate financial power to a few people if the app is successful and gains popularity. If it were to gain popularity, I would be the only one receiving money from all the people using my app if it has ads. If I were to hire more people, it still would be concentrated in only a few people. Even in that. case, I would not need to hire many people so the monetary gains would exceed expenses. Although it could be argued that my app would distribute power of time and free will if it were to be successful, it is not guaranteed that because it is popular, it is actually helping people be mindful. 


My app also faces the ethical question of whether making an app to help people be mindful is an effective solution to improving people’s mental health. This is important to consider because maybe an app is not the proper solution to helping people be mindful. In fact there is a significant amount of research that shows that there is little evidence is available on the efficacy of apps in developing mindfulness. Many apps exist that claim to be mindfulness-related. However, most are guided meditation apps, timers, or reminders and very few have high ratings on scores of visual aesthetics, engagement, functionality or information quality\cite{Mani2015ReviewAE}. Despite this evidence, there is also evidence that practicing mindfulness using a smartphone app may provide immediate positive effects on mood and stress while also providing long-term benefits for attention control. Because of this I am hopeful that a well designed mindfulness app could be an effective technology solution to develop mindfulness\cite{Walsh2019EffectsOA}.


The last ethical concern that I am aware of for my app is the issue of consent. I believe that my app could have issues with underage users because many parents may not want their underage children to use mindfulness apps. Many parents have had issues with kids practicing meditation because they believe it is going against their religion. Because of the ease with which kids could download apps marked as for adults only, this may cause problems with parents because they would argue that they did not give their kids permission to engage with this kind of content. At several schools, parents have spoken out that mindfulness practices violate their own religious beliefs. Parents were upset because mindfulness practices began amongst students without the parents knowing. They did not want their kids learning this content. 
\\
\end{body}


\begin{conclusion}

To conclude, my app definitely has several ethical issues that must be taken into consideration. It is crucial that I include community members in the design process so that I can have the minimum amount of data bias as possible. Additionally, by including data members in the design process, specifically people like Unaccompanied Migrant Youth, I would be able to make my app more accessible for them and for other people that I may not even be considering. There are definitely some unavoidable accessibility issues such as the fact that only  people with iPhones, iPads, or iPods could use my app and they must be able to understand English. However, if my app gains popularity I may have more resources at hand to be able to make the app support multiple languages and maybe expand to other platforms. Although the app may be able to be more accessible to people, the caveat of the app gaining popularity would be that it would then concentrate economic power to only the app designers. It would be hard to avoid this unless money is given directly to our users. Additionally, it is not clear whether or not an app for mindfulness is even an effective solution to helping people achieve mindfulness. However, there is evidence it has potential to be effective so I will design my app to follow proven effective designs. The last concern would about parents not being okay with their kids using this app. However, this is not an issue I can solve myself as many apps have the issue of being able to verify the age of users. So unfortunately I will not be able to combat this. Overall, the app has potential to be very ethical as it is trying to help improve the mental health of people. I have hope I can have my app be rid of most potential ethical issues. 

\end{conclusion}

\printbibliography 

\end{document}
